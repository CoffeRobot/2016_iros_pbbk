\section{Related Work}
\label{sec:relatedwork}

Assessing the performance of various computational techniques and providing the suitable benchmarks for these receives increasing attention in several areas  \cite{christensen02} COMMENT\footnote{We need more references here and more recent ones  benchmarks in robotics etc}. The need for evaluating feature descriptors in terms of object recognition, classification and alike is clearly important. One of the most relevant work has been done more than a decade ago by Mikolajczyk et al. \cite{mikolajczyk05} where affine region detectors are evaluated in terms of repatability and accuracy. SIFT feature descriptor \cite{lowe04} and its extension GLOH \cite{mikolajczyk05} showed the best performance on the dataset decribed in \cite{mikolajczyk2005b}. More recently, due to a stable increase of portable devices and large scale databases, there is more focus being 
put on the computational cost and runtime requirements of such descriptors. For example, SURF \cite{bay2008} is among the first descriptors that presented lower computational requirements than SIFT and comparable precision. More recently, Calonder et al. \cite{calonder10} proposed to use a binary descriptor BRIEF, as opposed to SIFT histogram, in combination with a fast corner detector FAST \cite{rosten06} achieving a fraction of the runtime performance to SURF. However, the descriptor has comparable results only when small transformations are applied to an image since it is not scale or rotation invariant. 

The ORB descriptor (Oriented Fast and Rotated Brief) proposed by Rublee et al. \cite{rublee11} enhances BRIEF adding the rotation invariance. Another binary descriptor, BRISK \cite{leutenegger11}, provides both scale and orientation invariance. Its corner detector, AGAST \cite{mair2010}, is also faster also faster than FAST. However, its invariance properties influence the overall performance.
Heinly et al. \cite{heinly2012} conducted an extensive evaluation of these three binary descriptors using SURF and SIFT as a baseline on the Oxford dataset \cite{mikolajczyk2005b}. As expected, SIFT demonstrated the most precise performance while BRIEF was the fastest one. However, since BRIEF is not rotation and scale invariant, its performance is significantly inferior in cases of significant image transformations. This just supports the fact that optimizing between precision and speed is not so trivial.  In the area of robotics, the low computational cost of an employed feature descriptors may have the high impact in real-time applications such as visual servoing, simultaneous mapping and localization (SLAM), and alike. 

Recently, a promising binary feature descriptor, KAZE, has been proposed by Alcantarilla et al. \cite{alcantarilla12}. The main strength of the descriptor lies in using non linear diffusion filtering techniques \cite{weickert98} to build the scale space as opposed to Gaussian blurring. Since the latter smooths without distinction at any scale, it does not preserve object boundaries. On the other hand, non-linear diffusion filtering techniques preserve edges, resulting in an increased distinctiveness of the feature descriptor. Once again, an improved precision comes at a cost of speed. However, by using the Fast Explicit Diffusion \cite{goesele2010}, an accelerated version of the feature descriptor AKAZE \cite{alcantarilla13} was proposed. AKAZE achieves better accuracy than SIFT and has a lower computational cost. The experiments have been performed on the Oxford dataset, designed to evaluate the precision of feature descriptors under various image transformations. Such feature descriptors are now employed in real-time system such as tracking by detection \cite{nebehay2014,pieropan15,pieropan15b} and  SLAM \cite{murartal2015}. 

