\section{Related Work}
\label{sec:relatedwork}

Evaluating the performance has a crucial role in computer vision to test the reliability of the algorithms and determine if they are of any practical use \cite{christensen02,butler12,wu2013}. Evaluating feature descriptors is not an exception. One of the most relevant work has been done in 2005 by Mikolajczyk et al. \cite{mikolajczyk05} where affine region detectors are evaluated in terms of repeatability and accuracy. SIFT feature descriptor \cite{lowe04} and its extension GLOH \cite{mikolajczyk05} had the best results on the dataset described in \cite{mikolajczyk2005b}. In recent years, due to a steady increase of portable devices and large scale databases, the runtime requirements of such descriptors have drawn more attention. SURF \cite{bay2008} is among the first descriptors that presented lower computational requirements than SIFT and comparable precision. More recently, Calonder et al. \cite{calonder10} proposed to use a binary descriptor BRIEF, as opposed to gradient histograms used in SIFT, in combination with a fast corner detector FAST \cite{rosten06} achieving a fraction of the runtime performance of SURF. However, the descriptor has comparable results only when small transformations are applied to an image, since it is not scale or rotation invariant. ORB descriptor (Oriented Fast and Rotated Brief) proposed by Rublee et al. \cite{rublee11} enhances BRIEF adding the rotation invariance. Another binary descriptor, BRISK \cite{leutenegger11}, provides both scale and orientation invariance. Its corner detector, AGAST \cite{mair2010} , is also faster than FAST however its invariance properties influence the overall performance.\\ 
Heinly et al. \cite{heinly2012} evaluated exhaustively these three binary descriptors using SURF and SIFT as baseline on the Oxford dataset \cite{mikolajczyk2005b}. Not surprisingly, SIFT was the most precise while BRIEF was the fastest one but, since it is not rotation and scale invariant, its performance is significantly lower than the other descriptors upon large image transformation. This evidence underlines the challenge in finding a good compromise between precision and speed. Nevertheless, the small computational requirements of such descriptors drawn the attention on their potential use in real-time applications such as tracking or simultaneous mapping and localization (SLAM). \\
A promising feature, KAZE, has been proposed by Alcantarilla et al. \cite{alcantarilla12}. Its main difference compared to other descriptors is the usage of non-linear diffusion filtering techniques \cite{weickert98} to build the scale space as opposed to Gaussian blurring. Since the latter smooths without distinction at any scale, it does not preserve object boundaries. On the other hand non-linear diffusion filtering techniques preserve edges, resulting in an increased distinctiveness of the feature descriptor. Once again, an improved precision comes at a cost in speed, however by the use of Fast Explicit Diffusion \cite{goesele2010} the accelerated version of the feature descriptor AKAZE \cite{alcantarilla13} the descriptor could achieve better accuracy than SIFT. Yet the experiments has been performed on the Oxford dataset, designed to evaluated the precision of feature descriptors with image transformation. Such features are now employed in real-time system such as tracking by detection \cite{nebehay2014,pieropan15,pieropan15b} or more in general SLAM \cite{murartal2015}. 

