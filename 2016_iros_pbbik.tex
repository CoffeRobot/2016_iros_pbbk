%%%%%%%%%%%%%%%%%%%%%%%%%%%%%%%%%%%%%%%%%%%%%%%%%%%%%%%%%%%%%%%%%%%%%%%%%%%%%%%%
%2345678901234567890123456789012345678901234567890123456789012345678901234567890
%        1         2         3         4         5         6         7         8

\documentclass[letterpaper, 10pt, conference]{ieeeconf}                             
\IEEEoverridecommandlockouts         % Needed if you want to use the \thanks command
                                                     
\overrideIEEEmargins                                                            


\usepackage{times} % assumes new font selection scheme installed
\usepackage{amsmath}
\usepackage{amssymb,longtable,calc}
\usepackage{mathptmx}
\usepackage[T1]{fontenc}                                                        
\usepackage[utf8]{inputenc}                                                     
\usepackage[english]{babel}                                                     
\usepackage{epsfig}                                                             
\usepackage{subfigure}                                                          
\usepackage{textcomp} %<- allows to use \textdegree but may overwrite           
                      %other settings                                           
\usepackage[textwidth=2cm,colorinlistoftodos]{todonotes} %add disable   
                                %to not show the todos                          
\usepackage{tikz}                                                               
\usetikzlibrary{arrows,positioning,fit,shapes,calc}
\usetikzlibrary{matrix}
\usepackage{flushend}                                                           
\usepackage{hyperref}  
\usepackage{amsmath}    

\usepackage[utf8]{inputenc}   
\usepackage[]{algorithm2e}
\usepackage{amsmath}


\usepackage{pgfplots} 
\usepackage{pgfplotstable}

\usepackage{cite}

% helper packages to work on the draft
\usepackage[tikz]{bclogo}
\usepackage{lipsum}

\usepackage{standalone}

\pgfplotsset{compat=newest}
\pgfplotsset{ 
  tick label style={font=\footnotesize}, 
  label style={font=\footnotesize}, 
  legend style={font=\footnotesize},
  title style = {font=\small}
}
\pgfplotscreateplotcyclelist{line style}{% 
  solid, mark options = {scale = .75}, every mark/.append style={fill=gray},mark=*\\% 
  densely dashed,mark options = {scale = .75},every mark/.append style={solid,fill=gray},mark=*\\% 
  densely dotted,mark options = {scale = .75},every mark/.append style={solid,fill=gray},mark=*\\% 
  dashed,mark options = {scale = .75},every mark/.append style={solid,fill=gray},mark=*\\% 
  dotted,mark options = {scale = .75},every mark/.append style={solid,fill=gray},mark=*\\% 
}
\pgfplotscreateplotcyclelist{bar style}{% 
  solid, fill=black!60!white\\%
  solid, fill=black!45!white\\%
  solid, fill=black!35!white\\%
  solid, fill=black!25!white\\%
}


\usepackage{xspace}
\makeatletter                                                                   
\DeclareTextCommandDefault{\textregisteredalt}{\footnotesize\textcircled{%      
      \check@mathfonts\fontsize\sf@size\z@\math@fontsfalse\selectfont R}}       
\DeclareRobustCommand\onedot{\futurelet\@let@token\@onedot}                     
\def\@onedot{\ifx\@let@token.\else.\null\fi\xspace}                             
\def\eg{e.g\onedot}                                                             
\def\ie{i.e\onedot}                                                             
\def\vgl{see }                                                                  
\def\Fig{Fig\onedot }                                                           
\def\Tab{Tab\onedot }                                                           
\def\Eq{Eq\onedot }
\def\Sec{Sec\onedot}                                                            
\def\etc{etc\onedot}                                                            
\def\etal{\textsl{et al}\onedot}                                                
\def\argmin{\mathop{\rm arg\,min}}                                              
\makeatother
                                                                    
\definecolor{lightGray}{rgb}{0.0,0.0,0.0}
\definecolor{kthColor}{RGB}{26,84,166}
                                                                                
\title{\LARGE \bf Feature Descriptors for Tracking by Detection: a Benchmark}                                                                               
                                                                                
                                                                                
\author{Alessandro Pieropan ~~~~ Mårten Bj{\"o}rkman  ~~~~ Niklas Bergstr{\"o}m ~~~~ Danica Kragic%
\thanks{This research has been supported by he Japan Society for the Promotion of Science (JSPS)}
\thanks{The GPU used for this research was donated by the NVIDIA Corporation.}
\thanks{MB and DN are with CVAP/CAS, KTH, Stockholm, Sweden, {\tt celle,dani@kth.se}. AP and NB are with the University of Tokyo, Japan, {\tt alessandro\_pieropan}, {\tt niklas\_bergstrom@ipc.i.u$-$tokyo.ac.jp}.}}

\begin{document}                                                                
                                                                                
\maketitle                                                                      
\thispagestyle{empty}                                                           
\pagestyle{empty}



%%%%%%%%%%%%%%%%%%%%%%%%%%%%%% ABSTRACT %%%%%%%%%%%%%%%%%%%%%%%%%%%%%%%%%%%%%
\begin{abstract}
In this paper, we provide an extensive evaluation of the performance of local descriptors for tracking applications.
Many different descriptors have been proposed in the literature for a wide range of application in Computer Vision such as object recognition and 3D reconstruction. More recently, due to fast key-point detectors, feature descriptor can be used in online tracking frameworks. However, while much effort has been spent on evaluating their performance in terms of distinctiveness and robustness to image transformations, very little has been done in the contest of tracking. Our evaluation is performed in terms of distinctiveness, tracking precision and tracking speed. Our results show that binary descriptors like ORB or BRISK have comparable results to SIFT or AKAZE due to a higher number of key-points.    

\end{abstract}

%%%%%%%%%%%%%%%%%%%%%%%%%%%%%% INTRODUCTION %%%%%%%%%%%%%%%%%%%%%%%%%%%%%%%%%%%
% \begin{}
\section{INTRODUCTION}
\label{sec:introduction}



Local regions of interest or key-point descriptors are widely used in Computer Vision for application such as object recognition and retrieval, 3D reconstruction and motion tracking. SIFT \cite{lowe04} is widely considered as one of the most robust feature descriptors, providing distinctiveness and invariance to common image transformations such as rotation and scale. However such a robustness comes with a computational cost. 

Recently there is much concern about efficiency caused mainly by two important factors. First there is a stable growth of portable camera enabled devices with limited computing power. Second, computer vision databases are steadily increasing in size. As a consequence, there is a growing interest within the computer vision community on fast key-point detectors and binary descriptors that can dramatically decrease the computational cost of detecting and matching local regions of interest. BRIEF \cite{calonder10} feature descriptor in combination with FAST \cite{rosten06} key-point detector is among the first attempts in this direction making it suitable for real time applications. However, despite the improvement in performance, BRIEF descriptor is not very robust to image transformations. This underlines the difficulty in finding a good compromise between two competing characteristics: distinctiveness and fast computation. 

\begin{figure}[t]
	\vspace{2mm}
\centerline{%
	\subfigure{\includegraphics[width=0.48\linewidth]{imgs/intro/intro1.png}}
	%\centerline{%
	\subfigure{\includegraphics[width=0.48\linewidth]{imgs/intro/intro3.png}}}
	\vspace{-2mm}
\centerline{%
	\subfigure{\includegraphics[width=0.48\linewidth]{imgs/intro/intro2.png}}
	\subfigure{\includegraphics[width=0.48\linewidth]{imgs/intro/intro4.png}}}
\caption{Examples showing some of the videos the descriptors has been tested on and the tracking results expressed as a coloured bounding box. Clearly some feature descriptors show more precise tracking.}
\vspace{-3mm}
\label{fig:intro}
\end{figure}

More attempts has been done in this direction. BRISK \cite{leutenegger11} and ORB \cite{rublee11} propose some modification of the BRIEF and FAST in order to achieve scale and rotation invariance. The mentioned descriptors are faster that SIFT and have comparable matching precision with small image transformation. More recently the binary feature descriptor KAZE \cite{alcantarilla12} presented comparable results to SIFT on a standard dataset \cite{mikolajczyk05} designed to evaluate the robustness of local descriptors on several image transformations. Moreover, by using fast non linear filtering techniques Fast Explicit Diffusion \cite{goesele2010}, its accelerated version AKAZE \cite{alcantarilla13} could also beat SIFT in computational cost.\\
The main reason why AKAZE could outperform SIFT relies in the use of non linear filtering techniques. However, given the increasing use of GPUs in computer vision, it is not clear if such techniques can easily parallelizable on a GPU architecture as opposed to Gaussian filtering employed in SIFT. A recent work \cite{jiang2015} has shown that it is possible to deploy AKAZE descriptor on a specialized hardware and achieve a good speedup compared to the original implementation. By implementing the descriptor in CUDA we intend to analyze its performance using a GPU and compare it against the GPU implementations of their counterpart descriptors.

%\missingfigure[figwidth=0.98\linewidth]{Intro figure showing tracker benchmark}

These extremely fast local descriptors not only improve the performance of vision tasks such as object retrieval and 3D reconstruction but they can be used in real-time tracking system. Yet, to the best of our knowledge, very little work has been done in evaluating key-points descriptor for the specific purpose of tracking. This work aims to target that issue by providing a fair and comprehensive evaluation of the most well known descriptors. We use the recall-precision measure to evaluate the matching distinctiveness of the features and we calculate the tracking precision by integrating each local descriptor in a key-point based tracker \cite{pieropan15}. Since there is no dataset designed to test local descriptors for tracking purposes , the videos used in the evaluation are gathered from well known tracking datasets described in \cite{wu2013,nebehay2014,hare2011}. We want to provide a practical guideline to binary descriptors for the specific task of tracking since it is not clear yet that a descriptor designed for image recognition is well suited for tracking.
 
The contribution of this work is four-fold:\\
\textbf{Dataset.} We collected 47 sequences from different public available datasets and annotated the ground-truth in a standardized format to ease the evaluation.\\
\textbf{Cuda Akaze.} Given the growing interest in the AKAZE feature descriptor and the lack of a GPU implementation, we provide our own using CUDA.\\
\textbf{Evaluation Library.} We integrated the most well known descriptors in a state-of-the-art key-point based 2D tracker and we provide an interface to enable the integration of new feature descriptors.\\
\textbf{Evaluation.}  There are three criteria the local descriptors are tested upon: distinctiveness, tracking precision and speed. First we measure the distinctiveness by matching the feature descriptors extracted in the first frame of the sequence and by computing the recall-precision. Then we calculate the tracking by integrating each feature descriptor in a key-point based tracker and we calculate the well known overlap accuracy for low, medium and high precision requirements. Last we profile the performance of each descriptor.

The dataset, our AKAZE implementation and all the code to perform the benchmark will be publicly available.



%% %%%%%%%%%%%%%%%%%%%%%%%%%%% RELATED WORK %%%%%%%%%%%%%%%%%%%%%%%%%%%%%%%% 
\section{Related Work}
\label{sec:relatedwork}

AKAZE: By means of nonlinear diffusion, we can increase repeatability and distinctiviness when detecting and describing an image region at different scale levels through a nonlinear scale space. 

In 2005, Mikolajczyk et al. [9] evaluated affine region detectors, and looked to define the repeatability and accuracy of several affine covariant region detectors

%% %%%%%%%%%%%%%%%%%%%%%%%%%%% DATASET AND EVALUATION DESCRIPTION %%%%%%%%%%%%%%%%%%%%%%%%%%%%%%%%
%\input{dataset}

%% %%%%%%%%%%%%%%%%%%%%%%%%%%% BENCHMARK %%%%%%%%%%%%%%%%%%%%%%%%%%%%%%%%
\section{Benchmark}

The purpose of the experiments conducted are three-fold. First we want to measure the descriptiveness of the feature descriptors for matching purposes, this is crucial for the recovery of a tracker upon object loss. Second we measure the tracking accuracy by integrating each feature descriptor in our 2D tracker and compute the overlap measure using the estimated object position. Third we profile each separate step required in tracking by detection (key-point detection, descriptor computation and feature match) in order to evaluate the performance of the feature descriptors for real-time frameworks.

\subsection{Dataset}

\missingfigure[figwidth=0.98\linewidth]{Figure showing some example taken from the dataset}

There are many publicly available datasets designed for tracking, however there is disagreement on how the data are stored. In order to facilitate the evaluation we collected the videos of different datasets and standardized how the data are stored. Each video is stored as a sequence of images while the ground truth, represented by an oriented bounding box, is saved in a comma separated value file where each row correspond to an image frame and contains 8 values representing the pairs x,y of each vertex of the box. The dataset will be released publicly, along with all the code to perform the experiments and compute the evaluation.

\subsection{Matching Evaluation Criteria}

\missingfigure[figwidth=0.98\linewidth]{Figure explaining how the matching ratios are calculated}

Given as sequence of images $I_{1},...,I_{n}$ and the bounding box ground truth $gb_{1},...,gb_{n}$, we extract the set of target key-points descriptors $K_{1}$ from the first image of the sequence and we label all the key-points within $gb_{1}$ as descriptors of the object. For any subsequent image key-points $K_{t}$ are extracted and matched with $K_{1}$, generating a list of matches $M(i,j)_{k}$, where \textit{i} indicates a key-point descriptor of our target set $K_{1}$, and \textit{j} a descriptor of our test set. the test set is then labeled as follows:

\begin{equation}
K_{t}^{j} = 
\begin{cases}
\text{true positive}  \text{ if } K_{t}^{j} \in gb_{n} \land K_{1}^{i} \in gb_{1} \\
\text{false positive}  \text{ if } K_{t}^{j} \notin gb_{n} \land K_{1}^{i} \in gb_{1} \\
\text{false negative}  \text{ if } K_{t}^{j} \in gb_{n} \land K_{1}^{i} \notin gb_{1} \\
\end{cases}
\end{equation}

where the pair \textit{i,j} is determined by the matching result $M(i,j)_{k}$. The average ratio of true positives underlines the ability of a tracker to find the object and potentially recover from track loss. False negatives are feature descriptors that appear in the current frame but have no corresponding match in the initial test set $K_{1}$. This may happen due to drastic change in appearance of the object. The ratio of false positives is very important to consider since it indicates the average number of outliers that will be used to estimate the pose of the object, resulting in a bad pose estimation without employing additional filtering techniques. One widely used filtering technique consists in discarding all matched key-points if the ratio between the score of the best match and the second best match is below a certain threshold $\rho$. We define a key-point as ambiguous, if this criteria is not met. The  number of ambiguous true and false positives is then calculated to evaluate the distinctiveness of the descriptors and evaluate the influence of this common filtering technique on the results.

\subsection{Evaluating tracking precision}

\begin{algorithm}[h]
 \KwData{$I_{1},...,I_{n},b_{1}$}
 \KwResult{$b_{2},...,b_{n}$}
 $K_{1} \gets$ \textbf{extract\_points}($I_{1},b_1$)\;
 \For{$i \gets 2 : n$}{
   $K_{i}^{*} \gets$ track\_points($K_{i-1},I_{i-1},I_i$)\;
   $b_i \gets$ estimate\_pose($K_{i}^{*}$)\;
   $K_{i}' \gets$ \textbf{extract\_points}($I_{i},b_i$)\;
   $M \gets$ \textbf{match\_points}($K_{1},K_{i}'$)\;
   $K_{i} \gets$ merge\_keypoints($K_i^* ,  M$)\;
 }
 \caption{\label{alg:algorithm}Overview of the tracking algorithm used to compute the precision. The feature descriptors are employed in the steps in bold. }
\end{algorithm}

In order to measure the precision of the feature descriptors for tracking we employ a sparse key-point based tracker. The tracker requires a bounding box in the initial image of a sequence as initialization to extract feature descriptors that represent the \textit{model} of the object to track. The algorithm estimates the position of the object, represented as an oriented bounding box, with a combination of sparse optical flow and feature matching as shown in Alg.~\ref{}. The original algorithm used ORB. We upgraded our algorithm making it more modular and able to cope with any possible feature descriptor. To estimate the precision we used the widely accepted overlap measure:

\begin{equation}
	\Theta (b_{t}, b_{gt}) = \frac{b_{t} \cap b_{gt}}{b_{t} \cup b_{gt}}
\end{equation}

where \textit{$b_{t}$} is the bounding box estimated by our tracker and
\textit{$b_{gt}$} is the bounding box provided by the ground truth. We define 
three precision requirements $\Upsilon$ (0.25, 0.5, 0.75) that indicates low, medium and high tracking accuracy. This is a more indicative evaluation compared to the overall accuracy. For instance, an overall value of 0.5 is ambiguous because it may indicate either a stable average accuracy around the value or a very precise evaluation in part of the video while poor in the rest.

The estimated object box \textit{$b_{t}$} is considered a true positive (TP) for a defined threshold of
accuracy $\Upsilon$ if:

\begin{equation}
\begin{cases}
b_{t} = TP  \text{ if } \Theta(b_{t}, b_{gt}) > \Upsilon \\
b_{t} = FP  \text{ otherwise }\\
\end{cases}
\end{equation}

The overall accuracy of the tracker for each precision requirement is calculated as:

\begin{equation}
\text{recall } = \frac{TP}{\text{TP } + \text{ FN}}
\end{equation}

\missingfigure[figwidth=0.98\linewidth]{Figure showing the estimated bounding boxes and the ground truth to show the different behavior of trackers}

\subsection{Evaluating tracking precision}

\begin{figure}
	%\vspace{-2mm}
	\includegraphics[width=0.95\linewidth]{imgs/performances.pdf}
\vspace{-2.5mm}	
\caption{Performance of the compute,detect and match steps of each feature descriptor.}
\label{fig:ros}
\end{figure}

\subsection{Descriptor performances}

\begin{figure*}
\centerline{%
	\subfigure{
		\includegraphics[width=0.5\linewidth]{imgs/brisk_correlation.pdf}~
		\includegraphics[width=0.5\linewidth]{imgs/orb_correlation.pdf}~
		}}
\vspace{-2.0mm}
\centerline{%
\subfigure{
		\includegraphics[width=0.5\linewidth]{imgs/akazec_correlation.pdf}~
		\includegraphics[width=0.5\linewidth]{imgs/sift_correlation.pdf}~
		}}
\vspace{-2.0mm}
\addtocounter{subfigure}{-1}
\centerline{%
	\subfigure{	
		\includegraphics[width=0.5\linewidth]{imgs/surf_correlation.pdf}~
		\includegraphics[width=0.5\linewidth]{imgs/cuda_sift_correlation.pdf}~
\label{fig:exp_d}		
}}
\vspace{-2.0mm}
\caption{Correlations matrices between the different evaluated measures of the feature descriptors.}
\vspace{-2mm}
\label{fig:depth_comparison_second}
\end{figure*}



\begin{figure*}[t]
\centerline{% 
		\includegraphics[width=0.98\linewidth]{tables/test.pdf}}
    \vspace{-2mm} 
	\caption{Overall results of the dataset for true positive matching. This is just a preview it won't be in he paper.}
	\label{fig:learnvsno}
\end{figure*}

%% %%%%%%%%%%%%%%%%%%%%%%%%%%% BENCHMARK %%%%%%%%%%%%%%%%%%%%%%%%%%%%%%%%
\section{Results}

In this section, we summarise the results of our assessment.

\subsection{Distinctiveness}

%\begin{figure*}[t]
%\centerline{% 
%		\includegraphics[width=0.98\linewidth]{imgs/distinctivenessTP.pdf}}
%    \vspace{-2mm} 
%	\caption{Examples taken from the dataset showing the ratio of true positives and ambiguous true positives. The lighter color bars show the number of true positives that will actually pass the second best result test.}
%	\label{fig:distinctiveness}
%\end{figure*}

The first aspect was to assess the distinctiveness. Table~\ref{table:tp_ratio} shows the average number of key-points extracted, descriptors belonging to the object to track and the ratio of true positives (TP), false positives (FP) and true positives that pass the second best match ratio check (TTP). 

\begin{table}[!h]
\caption{Average number of feature extracted, object features, true positives and false positives. Every row is normalized by its maximum value.}
\vspace{-2mm} 
\centerline{% 
		\includegraphics[width=0.98\linewidth]{tables/descriptivness_ratio.pdf}}
    \vspace{-2mm} 
	\label{table:tp_ratio}
\end{table}

It is interesting to notice that BRISK, ORB and SIFT extract a higher number of feature descriptors in general. In particular, BRISK and ORB have a higher number of key-points extracted within the area of the object. However, looking at the average amount of true positives, it can be seen that the best performing descriptors are AKAZE and the implementation of SIFT on the GPU. This is a first indicator of the quality of the descriptors extracted. Moreover, it can be noticed that the true positives are also more distinctive in the case of AKAZE and SIFT since the number of TTP is higher.

\begin{table*}[t]
\caption{Tracking results with low, medium and high accuracy requirements. The high number of key points extracted by ORB or BRISK compensate their weak descriptors. This comes with a cost in performance.} 
\centerline{%
		\includegraphics[width=\linewidth]{tables/tracking_precision.pdf}}
		\vspace{8mm}
	\label{table:taccuracy}
\end{table*}


\subsection{Tracking accuracy}

As explained in the previous section, we evaluated the performance of the feature descriptors by running our tracker and calculating the overlap measure for low, medium and high accuracy requirements. 
\begin{figure}[!h]
	\vspace{2mm}
\centerline{%
	\subfigure{\includegraphics[width=0.48\linewidth]{imgs/results/ex5.png}}
	\subfigure{\includegraphics[width=0.48\linewidth]{imgs/results/ex6.png}}}
\caption{Examples showing the behaviour of the feature descriptors upon occlusion. Upon recovery from track loss more descriptive descriptors allow the tracker to recover faster.}
\vspace{-3mm}
\label{fig:tracking_results}
\end{figure}

Table~\ref{table:taccuracy} summarizes tracking results on all the video sequences included in the dataset.  Our experiments show that AKAZE, BRISK, ORB and SIFT have comparable results. It is interesting to note that BRISK and ORB compensate their weak distinctiveness with a higher amount of weak descriptors extracted. A high number of feature points proved to be effective in tracking in the video sequences where the object suffers drastic scale changes and full occlusion, making the recovery after track loss faster, see Fig.~\ref{fig:tracking_results}. We also noticed that AKAZE, more than SIFT, suffers the change in scale. 

\subsection{Tracking performance}

The dataset used for benchmarking includes video sequences of various resolution. Fig.~\ref{fig:speed} shows the average performance of each feature descriptor on the resolutions having the highest number of sequences. 

\begin{figure}[!htb]
	%\vspace{-2mm}
	\includegraphics[width=0.95\linewidth]{imgs/tracker_fps_std.pdf}
\vspace{-2.5mm}	
\caption{Average time spent on tracking the object in a single frame: important factors are the resolution and the number of feature descriptors extracted. The variance of the results is a good indication of how much the number of feature descriptors influences the performance. It can be seen that the implementations have a lower variation due to the high level of parallelism. }
\label{fig:speed}
\end{figure}

The two most important factors that influence performance are resolution and number of key points extracted. The former influences particularly the detection step when the scale space of descriptors is computed and key-points are detected. The latter influences more the compute step when feature descriptors are calculated and the matching step. The average performance of each separate step can be seen in Fig.~\ref{fig:speed_b}. One interesting aspect to notice is the variance of the performance in Fig.~\ref{fig:speed}: BRISK, ORB and SIFT have the higher variance while CUDA SIFT and CUDA AKAZE have lower variance. This is also a good indicator of the level of parallelism of the implementation of the descriptor. It is interesting to notice that the computation of the non linear scale space required by AKAZE is not perfectly suited for a GPU architecture since it requires many sequential steps, as a result the key-detector is slower that SIFT. However since the descriptor its binary, its computation and matching compensate in terms of performance. It is also important to remember that CPU implementations do not exploit the same number of cores, therefore we think it is not fare to compare them qualitatively.

\begin{figure}[!htb]
	%\vspace{-2mm}
	\includegraphics[width=0.95\linewidth]{imgs/performances.pdf}
\vspace{-2.5mm}	
\caption{Performance of the compute, detect and match steps of each feature descriptor.}
\label{fig:speed_b}
\end{figure}

%\begin{table}
%\caption{Average time spent on a single frame by the tracker.}
	%\vspace{-2mm}
%	\includegraphics[width=0.95\linewidth]{tables/resolution_times.pdf}
%\vspace{-2.5mm}	
%\label{fig:fps}
%\end{table}

\subsection{Discussion}

Most of the feature descriptors proved to be effective for tracking purposes showing a good precision and performance. This is positive since it makes these suitable for real-time applications. Despite being assessed as somewhat weak in terms of distinctiveness, ORB and BRISK have been proven to be good for real time frameworks. 
However, the trade-off between the distinctiveness and accuracy is not easy to define.
AKAZE and SIFT have been proven to be more distinctive as descriptors but only their implementations on the GPU allow real time performances. 
The most important factors that diversify the results are the characteristics of the video sequences: object transformations, in particular scale, object appearance, light condition and motion blur. AKAZE and SIFT have proven to be more distinctive and effective when the video does not present drastic movements. AKAZE seems to be particularly sensitive to change in scale, on the other hand it has higher performance on low textured objects and people, still in sequences where the tracked object keep a constant distance from the camera.
Moreover, we noticed that the matching rate of all features drop consistently upon fast movements of the camera as we discussed in \cite{pieropan15}. However, the performance of weak feature descriptors such as BRISK or ORB seems to be less sensitive to this kind of noise. 

\begin{figure}[!htb]
	\vspace{2mm}
\centerline{%
	\subfigure[scale]{\includegraphics[width=0.33\linewidth]{imgs/limitations/scale.png}\label{fig:tra}}
	\subfigure[light]{\includegraphics[width=0.33\linewidth]{imgs/limitations/light.png}\label{fig:trb}}
	\subfigure[multi-instance]{\includegraphics[width=0.33\linewidth]{imgs/limitations/basket.png}\label{fig:trc}}}
	\vspace{-2mm}
\caption{Examples showing the main problems that feature descriptors cannot address. }
\label{fig:tracking_results_scale}
\end{figure} 


There are still some issue that need to be addressed in relation to achieving a robust tracking system. First, even if many descriptors are invariant to scale or rotation, we noticed that this does
not hold for drastic changes like in Fig.~\ref{fig:tra}. One possible solution consists in extracting features from appearances of the object generated through synthetic transformations. It has been shown by Morel \cite{morel2009} that this technique improves the matching performance of SIFT descriptor.  
Second, feature descriptors are sensitive to light conditions, see Fig.~\ref{fig:trb}. This is particularly relevant for robotics applications since the interaction of a robotic platform with a target object may occlude the light source. Third, the common matching approach to detect an object or compute the transformation between images \cite{mikolajczyk05} does not work in the presence of multiple targets with similar appearance. This is the case shown in Fig.\ref{fig:trc} where more players have the same outfit.











%% %%%%%%%%%%%%%%%%%%%%%%%%%%% CONCLUSION %%%%%%%%%%%%%%%%%%%%%%%%%%%%%%%%
\section{Conclusion}

\begin{figure}[b]
	\vspace{2mm}
\centerline{%
	\subfigure[scale]{\includegraphics[width=0.33\linewidth]{imgs/limitations/scale.png}\label{fig:tra}}
	\subfigure[light]{\includegraphics[width=0.33\linewidth]{imgs/limitations/light.png}\label{fig:trb}}
	\subfigure[multi-instance]{\includegraphics[width=0.33\linewidth]{imgs/limitations/basket.png}\label{fig:trc}}}
	\vspace{-2mm}
\caption{Examples showing the main problems that feature descriptors cannot address. }
\label{fig:tracking_results_scale}
\end{figure} 



We proposed an evaluation of the most common feature descriptors for the purpose of tracking by detection. Our experiments have shown that most of the feature descriptors have comparable results. While AKAZE and SIFT have proven to be more distinctive, ORB and BRISK compensate their weak descriptors with a higher number of points extracted. Given the growing interested in AKAZE descriptor we provided a GPU implementation so that it can be used for real time system. The code to perform the benchmark, the dataset and our implementation of AKAZE will be publicly available in order to ease researches in this area.



\begin{table*}[h]
\caption{Tracking results with low,medium and high accuracy requirements. The high number of key points extracted by ORB or BRISK compensate their weak descriptors. This comes with a cost in performance.} 
\centerline{%
		\includegraphics[width=0.98\linewidth]{tables/tracking_precision.pdf}}
    \vspace{-2mm} 
	\label{table:taccuracy}
\end{table*}

\bibliographystyle{unsrt}
\bibliography{ref}

%\section{Additional}

\todo[inline]{Please note this is a collection of tables and images that will not end in the paper. They are good anyway to have a picture of the results}

\begin{figure*}
\centerline{%
	\subfigure{
		\includegraphics[width=0.5\linewidth]{imgs/brisk_correlation.pdf}~
		\includegraphics[width=0.5\linewidth]{imgs/orb_correlation.pdf}~
		}}
\vspace{-2.0mm}
\centerline{%
\subfigure{
		\includegraphics[width=0.5\linewidth]{imgs/akazec_correlation.pdf}~
		\includegraphics[width=0.5\linewidth]{imgs/sift_correlation.pdf}~
		}}
\vspace{-2.0mm}
\addtocounter{subfigure}{-1}
\centerline{%
	\subfigure{	
		\includegraphics[width=0.5\linewidth]{imgs/surf_correlation.pdf}~
		\includegraphics[width=0.5\linewidth]{imgs/cuda_sift_correlation.pdf}~
\label{fig:exp_d}		
}}
\vspace{-2.0mm}
\caption{Correlations matrices between the different evaluated measures of the feature descriptors.}
\vspace{-2mm}
\label{fig:depth_comparison_second}
\end{figure*}

\begin{figure*}[t]
\centerline{% 
		\includegraphics[width=0.98\linewidth]{tables/test.pdf}}
    \vspace{-2mm} 
	\caption{Overall results of the dataset for true positive matching. This is just a preview it won't be in he paper.}
	\label{fig:learnvsno}
\end{figure*}

\end{document}
